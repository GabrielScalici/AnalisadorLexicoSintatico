\section{Como Usar \label{sec:como-usar}}

\subsection{Compilação}

O trabalho entregue, como requisitado, já foi previamente compilado (\texttt{Linux}), não havendo necessidade de executar esse passo. Porém, caso queira ou precise compilar novamente, basta estar dentro do diretório do trabalho e executar:

	\indent\indent\texttt{make}

É necessário ter instalado o compilador \texttt{gcc}, as ferramentas \texttt{flex} e \texttt{bison}, assim como o utilitário \texttt{make} em sistema operacional \texttt{Linux}.

\subsection{Execução}

Para executar o trabalho, basta estar dentro de seu diretório e executar:

	\indent\indent\texttt{./main}

Dessa maneira, o programa \texttt{LALG} será lido da entrada padrão \texttt{stdin}.

Para executá-lo sobre um arquivo, basta redirecionar a entrada:

	\indent\indent\texttt{./main < meu-programa.lalg}

No diretório \texttt{./test/sem} encontram-se alguns exemplos \textbf{semânticos} de programa em \texttt{LALG} para testar. Por exemplo:

	\indent\indent\texttt{./main < ./test/sem/programa1.lalg}

Opcionalmente, para rodar para todos os programas \texttt{.lalg} de \texttt{./test/sem}, execute:

	\indent\indent\texttt{make run}

As saídas serão escritas em arquivos com sufixo \texttt{\char`_out} na própria pasta \texttt{./test/sem}.

\newpage
\subsection{Exemplo de Execução}

Arquivo \texttt{./test/sem/error\char`_varios1.lalg} -- programa fictício com vários erros semânticos:
\begin{verbatim}
 1. program test1;
 2. 
 3.     { OK: declaracao de constantes }
 4.     const a = 10;
 5.     const b = 11;
 6.     const k = 238.11;
 7.     
 8.     { ERRO: constante ja declarada }
 9.     const a = 74.2;
10.     
11.     { OK: declaracao de variaveis}
12.     var x, y: real;
13.     var c, i, j: integer;
14.     var opcao1, opcao2: char;
15.     
16.     { ERRO: variavel ja declarada }
17.     var x: integer;
18. 
19.     { OK: declaracao de procedimento }
20.     procedure my_proc(x: integer; y, z: real);
21.         { OK: declaracao de variaveis }
22.         var i, j: integer;
23.     begin
24.         j := 7;
25.         for i:=1 to 5 do
26.         begin
27.             { ERRO: variavel 'c' nao declarada neste escopo }
28.             c := 5 * i + j * x + y + z;
29.         end;
30. 
31.     end;
32.     
33.     { ERRO: procedimento ja foi declarado }
34.     procedure my_proc(x: integer; y, z: real);
35.     begin
36.     end;
37.     
38. begin
39.     { OK: divisao entre inteiros }
40.     c := 439 / 2;
41. 
42.     { ERRO: tipo incompativel real <- integer }
43.     x := a;
44.     x := 25;
45. 
46. 
47.     { ERROS: divisao entre nao-inteiros (mas faz atribuicao) }
48.     y := 439.1 / 2;
49.     y := 500.0 / 2.0;
50.     
51.     { OK: atribuicoes sem problemas }
52.     opcao1 := 'A';
53.     opcao2 := '\n';
54.     c := a + b - 10;
55.     x := k * 2.0;
56.     x := y;
57.     x := 12.57 * 6.3;
58. 
59.     { ERRO: tipo incompativel integer <- real }
60.     c := x;
61.     c := 9.21;
62.     
63.     { OK: comandos sem problemas }
64.     read(x, y);
65.     write(x, y);
66. 
67.     { OK: atribuicoes sem problemas }   
68.     c := 9;
69.     i := c + a + b;
70.     j := i * 2;
71.     
72.     { ERRO: variavel nao declarada }
73.     t := x;
74.     
75.     { OK: chamada sem problemas }
76.     my_proc(c; x; y);
77.     
78.     my_proc;       { ERRO: argumentos insuficientes }
79.     my_proc(c);    { ERRO: argumentos insuficientes }
80.     my_proc(x);    { ERRO: argumentos insuficientes }
81.     my_proc(c; y); { ERRO: argumentos insuficientes }
82.     my_proc(c; x; y; j); { ERRO: argumentos demais }
83.     my_proc(x; y; j);    { ERRO: tipos incompativies }
84.     
85.     { ERRO: procedimento nao declarado }
86.     other_proc(x; y);
87.     
88.     { ERROS: comandos com variaveis de tipos diferentes }
89.     read(x, c);
90.     write(x, c);
91. end.\end{verbatim}

\newpage
Comando:

	\indent\indent\texttt{./main < ./test/sem/error\char`_varios1.lalg}

Saída:
\begin{verbatim}[ 9,19]: constant 'a' has already been declared
[17,19]: variable 'x' has already been declared
[28,39]: undeclared variable 'c'
[36,8 ]: procedure 'my_proc' has already been declared
[43,11]: incompatible type for variable 'x'
[44,12]: incompatible type for variable 'x'
[48,19]: division with non-integer numerator
[49,18]: division with non-integer denominator
[49,21]: division with non-integer numerator
[60,11]: incompatible type for variable 'c'
[61,14]: incompatible type for variable 'c'
[73,11]: undeclared variable 't'
[78,12]: insufficient number of arguments for procedure 'my_proc'
[79,14]: insufficient number of arguments for procedure 'my_proc'
[80,14]: insufficient number of arguments for procedure 'my_proc'
[81,17]: insufficient number of arguments for procedure 'my_proc'
[82,23]: too much arguments for procedure 'my_proc'
[83,20]: incompatible type in argument 1 for 'my_proc' procedure
[83,20]: incompatible type in argument 3 for 'my_proc' procedure
[86,20]: undeclared procedure 'other_proc'
[89,14]: read/write command with different variable types
[90,15]: read/write command with different variable types\end{verbatim}

Onde \texttt{[i,j]} indica linha \texttt{i} na coluna {\texttt{j}}.

Mais exemplos estão disponíves no diretório \texttt{./test/sem}.
