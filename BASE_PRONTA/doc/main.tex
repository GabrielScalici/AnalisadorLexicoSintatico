% Inicializa o documento
% define papel, tamanho global de fonte, tipo de documento
\documentclass[a4paper, twoside, 12pt]{article}

	% Multicolunas
	\usepackage{multicol}

	% Pacotes usados
	\usepackage[utf8]{inputenc} % enconding de caracteres
	\usepackage[brazil]{babel}  % locale pt_BR
	\usepackage[lmargin=2cm, rmargin=2cm, tmargin=2cm, bmargin=2cm]{geometry} % margens da folha
	\usepackage{indentfirst} % sempre indenta o primeiro parágrafo

	% Matemática	
	\usepackage{amsmath}
	\usepackage{amsfonts}
	\usepackage{gensymb}
	
	% Tabela longa que quebra entre as paginas
	\usepackage{longtable}

	% Para links e url
	\usepackage[hidelinks]{hyperref}
	
	\usepackage{listings} % listagem de código-fonte
	\renewcommand*{\lstlistingname}{Listagem} % texto para listagem de código
	\usepackage{color} % cor para usar na listagem de código-fonte
%	\usepackage{svg}
	\usepackage{graphicx,xcolor} % para inserir imagens
	\usepackage[nottoc,notlot,notlof]{tocbibind} % adiciona o tópico Referências ao Sumário
	\usepackage{textcomp} % accesso \textquotesingle

	% Para desenhar grafos
	\usepackage{tikz}
	\usetikzlibrary{arrows,positioning,shapes,decorations}

	% Desenhar circulos
%	\usepackage{tkz-euclide}
%	\usetkzobj{all}

	% Escrever algoritmos em pseudo-código
%	\usepackage[portuguese,linesnumbered]{algorithm2e}

	% Tabelas
	\usepackage{booktabs}
	\usepackage{caption}

	% Estilos para usar nos grafos
	\tikzset{
		>=stealth',
		punkt/.style={
			rectangle,
			text centered,
			inner sep=0.7em,
			draw,
			fill=blue!5
		},
		pil/.style={
			->,
			thick,
			shorten <=2pt,
			shorten >=2pt
		}
	}

	% Para plotar gráficos
%	\usepackage{pgfplots}
%	\pgfplotsset{width=10cm,compat=1.9}

	% Define cores para o highlight de código-fonte
	\definecolor{dkgreen}{rgb}{0,0.6,0}
	\definecolor{gray}{rgb}{0.5,0.5,0.5}
	\definecolor{mauve}{rgb}{0.58,0,0.82}
	
	% Define configuração para listagem de código-fonte em linguagem C
	\lstset{
		frame=tb,
		language=C,
		aboveskip=2mm,
		belowskip=2mm,
		showstringspaces=false,
		columns=flexible,
		basicstyle={\small\ttfamily},
		numbers=none,
		keywordstyle=\color{blue},
		commentstyle=\color{dkgreen},
		stringstyle=\color{mauve},
		breaklines=true,
		breakatwhitespace=false,
		tabsize=4
	}

% Começo do documento
\begin{document}

	% Define algum espaçamento que eu não lembro, hehe :)
	\setlength\parskip{0.3cm}

	% Insere a Capa
	% Começo da folha de Capa
\begin{titlepage}

		% Título
		\title{
\textsc {\large{Universidade de São Paulo\\
Instituto de Ciências Matemáticas e de Computação}}\\[1cm]
\large{SCC0217 -- Linguagens de Programação e Compiladores}\\[6cm]
\LARGE{Trabalho 3 -- Analisador Semântico para LALG}\\[5.5cm]
		}

		% Autores
		\author{
Gabriel Rodrigues do Prado Rossales -- 6608843
		}

		% Inserção manual de data
		\date{
\vfill São Carlos, 15 de agosto de 2017
		}

		% Cria a Capa
		\maketitle
		\thispagestyle{empty}

% Fim da folha de Capa
\end{titlepage}

	
	% Reseta contador de página para 1 (assim não conta a Capa como página)
	\setcounter{page}{1}
	
	% Insere as outras partes do documento
	% Cria o Sumário
\tableofcontents

% Cria uma nova página, forçando o Sumário a ficar numa página separada
\newpage


	\section{Introdução \label{sec:introducao}}

Este trabalho implementa um analisador semântico com tratamento de erros para a linguagem de programação \texttt{LALG} utilizando as ferramentas \texttt{flex} e \texttt{bison}. Foram seguidas as instruções dadas em sala de aula assim como consultadas em manual~\cite{bib:manual} e em livro~\cite{bib:livro}.

	\section{Como Usar \label{sec:como-usar}}

\subsection{Compilação}

O trabalho entregue, como requisitado, já foi previamente compilado (\texttt{Linux}), não havendo necessidade de executar esse passo. Porém, caso queira ou precise compilar novamente, basta estar dentro do diretório do trabalho e executar:

	\indent\indent\texttt{make}

É necessário ter instalado o compilador \texttt{gcc}, as ferramentas \texttt{flex} e \texttt{bison}, assim como o utilitário \texttt{make} em sistema operacional \texttt{Linux}.

\subsection{Execução}

Para executar o trabalho, basta estar dentro de seu diretório e executar:

	\indent\indent\texttt{./main}

Dessa maneira, o programa \texttt{LALG} será lido da entrada padrão \texttt{stdin}.

Para executá-lo sobre um arquivo, basta redirecionar a entrada:

	\indent\indent\texttt{./main < meu-programa.lalg}

No diretório \texttt{./test/sem} encontram-se alguns exemplos \textbf{semânticos} de programa em \texttt{LALG} para testar. Por exemplo:

	\indent\indent\texttt{./main < ./test/sem/programa1.lalg}

Opcionalmente, para rodar para todos os programas \texttt{.lalg} de \texttt{./test/sem}, execute:

	\indent\indent\texttt{make run}

As saídas serão escritas em arquivos com sufixo \texttt{\char`_out} na própria pasta \texttt{./test/sem}.

\newpage
\subsection{Exemplo de Execução}

Arquivo \texttt{./test/sem/error\char`_varios1.lalg} -- programa fictício com vários erros semânticos:
\begin{verbatim}
 1. program test1;
 2. 
 3.     { OK: declaracao de constantes }
 4.     const a = 10;
 5.     const b = 11;
 6.     const k = 238.11;
 7.     
 8.     { ERRO: constante ja declarada }
 9.     const a = 74.2;
10.     
11.     { OK: declaracao de variaveis}
12.     var x, y: real;
13.     var c, i, j: integer;
14.     var opcao1, opcao2: char;
15.     
16.     { ERRO: variavel ja declarada }
17.     var x: integer;
18. 
19.     { OK: declaracao de procedimento }
20.     procedure my_proc(x: integer; y, z: real);
21.         { OK: declaracao de variaveis }
22.         var i, j: integer;
23.     begin
24.         j := 7;
25.         for i:=1 to 5 do
26.         begin
27.             { ERRO: variavel 'c' nao declarada neste escopo }
28.             c := 5 * i + j * x + y + z;
29.         end;
30. 
31.     end;
32.     
33.     { ERRO: procedimento ja foi declarado }
34.     procedure my_proc(x: integer; y, z: real);
35.     begin
36.     end;
37.     
38. begin
39.     { OK: divisao entre inteiros }
40.     c := 439 / 2;
41. 
42.     { ERRO: tipo incompativel real <- integer }
43.     x := a;
44.     x := 25;
45. 
46. 
47.     { ERROS: divisao entre nao-inteiros (mas faz atribuicao) }
48.     y := 439.1 / 2;
49.     y := 500.0 / 2.0;
50.     
51.     { OK: atribuicoes sem problemas }
52.     opcao1 := 'A';
53.     opcao2 := '\n';
54.     c := a + b - 10;
55.     x := k * 2.0;
56.     x := y;
57.     x := 12.57 * 6.3;
58. 
59.     { ERRO: tipo incompativel integer <- real }
60.     c := x;
61.     c := 9.21;
62.     
63.     { OK: comandos sem problemas }
64.     read(x, y);
65.     write(x, y);
66. 
67.     { OK: atribuicoes sem problemas }   
68.     c := 9;
69.     i := c + a + b;
70.     j := i * 2;
71.     
72.     { ERRO: variavel nao declarada }
73.     t := x;
74.     
75.     { OK: chamada sem problemas }
76.     my_proc(c; x; y);
77.     
78.     my_proc;       { ERRO: argumentos insuficientes }
79.     my_proc(c);    { ERRO: argumentos insuficientes }
80.     my_proc(x);    { ERRO: argumentos insuficientes }
81.     my_proc(c; y); { ERRO: argumentos insuficientes }
82.     my_proc(c; x; y; j); { ERRO: argumentos demais }
83.     my_proc(x; y; j);    { ERRO: tipos incompativies }
84.     
85.     { ERRO: procedimento nao declarado }
86.     other_proc(x; y);
87.     
88.     { ERROS: comandos com variaveis de tipos diferentes }
89.     read(x, c);
90.     write(x, c);
91. end.\end{verbatim}

\newpage
Comando:

	\indent\indent\texttt{./main < ./test/sem/error\char`_varios1.lalg}

Saída:
\begin{verbatim}[ 9,19]: constant 'a' has already been declared
[17,19]: variable 'x' has already been declared
[28,39]: undeclared variable 'c'
[36,8 ]: procedure 'my_proc' has already been declared
[43,11]: incompatible type for variable 'x'
[44,12]: incompatible type for variable 'x'
[48,19]: division with non-integer numerator
[49,18]: division with non-integer denominator
[49,21]: division with non-integer numerator
[60,11]: incompatible type for variable 'c'
[61,14]: incompatible type for variable 'c'
[73,11]: undeclared variable 't'
[78,12]: insufficient number of arguments for procedure 'my_proc'
[79,14]: insufficient number of arguments for procedure 'my_proc'
[80,14]: insufficient number of arguments for procedure 'my_proc'
[81,17]: insufficient number of arguments for procedure 'my_proc'
[82,23]: too much arguments for procedure 'my_proc'
[83,20]: incompatible type in argument 1 for 'my_proc' procedure
[83,20]: incompatible type in argument 3 for 'my_proc' procedure
[86,20]: undeclared procedure 'other_proc'
[89,14]: read/write command with different variable types
[90,15]: read/write command with different variable types\end{verbatim}

Onde \texttt{[i,j]} indica linha \texttt{i} na coluna {\texttt{j}}.

Mais exemplos estão disponíves no diretório \texttt{./test/sem}.

	\clearpage
\section{Organização dos Arquivos \label{sec:organizacao-dos-arquivos}}

O diretório do trabalho está organizado da seguinte maneira:

\indent\indent\texttt{./doc} : diretório dos arquivos \LaTeX~fonte deste relatório.

\indent\indent\texttt{./test} : diretório com exemplos de programa \texttt{LALG} para testes.\\
\indent\indent\indent\texttt{|-- ./lex} : exemplos léxicos.\\
\indent\indent\indent\texttt{|-- ./sem} : exemplos semânticos.\\
\indent\indent\indent\texttt{|-- ./sin} : exemplos sintáticos.

\indent\indent\texttt{\texttt{LALG}} : definição da linguagem \texttt{LALG}.

\indent\indent\texttt{Makefile} : arquivo para automizar compilação e execução usando o utilitário \texttt{make}.

\indent\indent\texttt{RELATORIO.pdf} : este relatório PDF compilado a partir de \texttt{./doc}.

\indent\indent\texttt{README} : arquivo com instruções.

\indent\indent\texttt{lalg.l} : programa \texttt{Lex} para a linguagem \texttt{LALG}.

\indent\indent\texttt{lalg.y} : programa em \texttt{Bison} para a linguagem \texttt{LALG}.

\indent\indent\texttt{lex.yy.c} : programa \texttt{C} gerado pelo \texttt{flex}.

\indent\indent\texttt{y.tab.c} : programa principal \texttt{C} gerado pelo \texttt{bison}.

\indent\indent\texttt{y.tab.h} : cabeçalho \texttt{C} gerado pelo \texttt{bison}.

\indent\indent\texttt{main} : o programa principal a ser executado para fazer a análise semântica.
	\section{Decisões de Projeto \label{sec:decisoes-de-projeto}}

\subsection{\texttt{lalg.y}}

Para armazenar os símbolos -- variáveis, constantes e procedimentos --, foram utilizadas tabelas \textit{hash} da biblioteca \texttt{<search.h>}. Uma tabela \textit{hash} é utilizada por escopo, ou seja, ao necessitar de um novo escopo, uma nova tabela é criada para armazenar os símbolos desse novo escopo.

Para simplificar a codificação, definiu-se estruturas e uniões de dados convenientes, como por exemplo, \texttt{struct data\char`_s} que guarda o tipo e o valor de um dado; e o valor de um dado é representado pela \texttt{union val\char`_u} que pode assumir qualquer um dos possíveis valores do LALG.

A detecção dos erros semânticos consistiu em manipular as tabelas \textit{hash} e estruturas de dados auxiliares como listas temporárias de variáveis, parâmetros e argumentos.

\subsection{\texttt{lalg.l}}

No parte léxica foi apenas necessário alterar a os retornos dos valores dos \textit{tokens}, que agora utilizam a \texttt{struct data\char`_s}.

	\section{Conclusão \label{sec:conclusao}}

O trabalho desenvolvido cumpre a especificação dada. Foi possível aprender mais sobre a ferramenta \texttt{bison} e concluir o analisador semântico de \texttt{LALG} que usou o analisador sintático e léxico implementados nos trabalhos anteriores.

	% Começo das Referências
\begin{thebibliography}{9}

% EXEMPLO DE REFERENCIA
%	\bibitem{bib:open-mpi}
%		Open MPI\\
%		\textless\url{http://www.open-mpi.org/doc/v1.8/}\textgreater\\
%		Acesso em: 6 de novembro de 2014.

	\bibitem{bib:manual}
		Bison 3.0.4 \\
		\textless\texttt{http://www.gnu.org/software/bison/manual/html\char`_node/index.html}\textgreater\\

	\bibitem{bib:livro}
		LEVINE, John. flex \& bison. United States of America: O'Reilly, 2009.


% Fim das Referências
\end{thebibliography}

% http://upload.wikimedia.org/wikipedia/commons/8/8f/Whole_world_-_land_and_oceans_12000.jpg


% Fim do documento
\end{document}
